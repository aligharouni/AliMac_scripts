\documentclass[12pt]{article}
\usepackage{natbib}
\usepackage{hyperref}
\usepackage{graphicx}
\usepackage{subcaption}
\usepackage{amssymb,amsmath,amsthm}
\usepackage{xcolor}
\usepackage{xspace}
\usepackage[nameinlink,capitalize]{cleveref}
\usepackage{cleveref}
\usepackage[margin=1in]{geometry}
\usepackage{lineno}\renewcommand\thelinenumber{\color{gray}\arabic{linenumber}}
\usepackage{pdflscape}
\usepackage{xspace}
\usepackage{array}

\newcolumntype{L}[1]{>{\raggedright\let\newline\\\arraybackslash\hspace{0pt}}m{#1}}
\newcolumntype{C}[1]{>{\centering\let\newline\\\arraybackslash\hspace{0pt}}m{#1}}
\newcolumntype{R}[1]{>{\raggedleft\let\newline\\\arraybackslash\hspace{0pt}}m{#1}}

\newcommand{\comment}{\showcomment}
\newcommand{\showcomment}[3]{\textcolor{#1}{\textbf{[#2: }\textsl{#3}\textbf{]}}}
\newcommand{\nocomment}[3]{}

\newcommand{\ali}[1]{\comment{magenta}{Ali}{#1}}
\newcommand{\bmb}[1]{\comment{red}{BMB}{#1}}
\newcommand{\todo}[1]{\comment{red}{TODO}{#1}}

\theoremstyle{definition} % amsthm only
\newtheorem{proposition}{Proposition}
\newtheorem{theorem}{Theorem}

\bibliographystyle{apalike}

\title{Centrality: A Curved-Based Statistical Discription of an Ensemble}
\author{Ali Gharouni, Ben Bolker}
\begin{document}
\maketitle
\linenumbers

% %%%%%%%%%%%%%%%%%%%%%%%%%%%%%%%%%%%%%%%
\section{Introduction}

\bibliography{../CentLib}

% %%%%%%%%%%%%%%%%%%%%%%%%%%%%%%%%%%%%%%%
\section{Appendix}

\subsection{Literature Review}
\citep{juul2021fixed}

\subsection{emails}

{\bf Ben,  
date: Feb 9, 2021}

There are several somewhat orthogonal questions, all geared to the general question of "what is a sensible way to define a confidence interval or 'typical' set of curves out of an ensemble?

 1. Do we apply measures of centrality directly to the curves or to
probes derived from the curves?
 2.  What measure of centrality/typical-ness do we apply?
 3.  If it depends on distance, what distance do we use?
 4. Do we define centrality by difference from a central point
(centroid/Fréchet mean) or by average closeness to the rest of the
set? Does the answer differ depending on step 3 (choice of distance)?

(And finally: how do Juul et al's methods fit into this?)

Another email the same day:
you might be able to show that the conjecture does *not* hold for some fairly simple distance metric (e.g. log(Euclidean) or some similar kind of geometric-mean distance), in which case it would also settle most of the question.





\end{document}