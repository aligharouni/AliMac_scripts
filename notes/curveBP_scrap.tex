\documentclass[12pt]{article}
\usepackage{natbib}
\usepackage{hyperref}
\usepackage{graphicx}
\usepackage{subcaption}
\usepackage{amssymb,amsmath,amsthm}
\usepackage{xcolor}
\usepackage{xspace}
\usepackage[nameinlink,capitalize]{cleveref}
\usepackage{cleveref}
\usepackage[margin=1in]{geometry}
\usepackage{lineno}\renewcommand\thelinenumber{\color{gray}\arabic{linenumber}}
\usepackage{pdflscape}
\usepackage{xspace}
\usepackage{array}

\newcolumntype{L}[1]{>{\raggedright\let\newline\\\arraybackslash\hspace{0pt}}m{#1}}
\newcolumntype{C}[1]{>{\centering\let\newline\\\arraybackslash\hspace{0pt}}m{#1}}
\newcolumntype{R}[1]{>{\raggedleft\let\newline\\\arraybackslash\hspace{0pt}}m{#1}}

\newcommand{\comment}{\showcomment}
\newcommand{\showcomment}[3]{\textcolor{#1}{\textbf{[#2: }\textsl{#3}\textbf{]}}}
\newcommand{\nocomment}[3]{}

\newcommand{\ali}[1]{\comment{magenta}{Ali}{#1}}
\newcommand{\bmb}[1]{\comment{red}{BMB}{#1}}
\newcommand{\todo}[1]{\comment{red}{TODO}{#1}}

\theoremstyle{definition} % amsthm only
\newtheorem{proposition}{Proposition}
\newtheorem{theorem}{Theorem}
 
\bibliographystyle{apalike}

\title{Centrality: A Curve-Based Statistical Discription of an Ensemble}
\author{Ali Gharouni, Ben Bolker}
\begin{document}
\maketitle
\linenumbers

% %%%%%%%%%%%%%%%%%%%%%%%%%%%%%%%%%%%%%%%
\section{Introduction}


\bibliography{../AliMac}

% %%%%%%%%%%%%%%%%%%%%%%%%%%%%%%%%%%%%%%%
\section{Appendix}


Goal: 
 How to compare Juul's result to ours, quantitative/visual? the point is that in addition to the option that Juul et al gave, here we present other options and you should think about which ones have the properties that you want.
 
What to be presented? 
- maybe 4 panels, here is what you get by various methods. Each panel have several quantile overlaid (50,80,90) 1 Juul's work, 1 Mahalanobis on the probes,  

Notes:
1) The central region (not the curve) is to be focused on. (spend a paragraph) 

2) What did Juul do? Algorithm? 
Out of the whole curves, referred as the whole ensemble, pick a random sensible, consider the pointwise min and max which determines the envelope. $E_{sample}$ is the envelope of the sample, assign a score to each curve in the ensemble, all scores starts with 0 if the curve is not in the envelope, they add $s(c_i)$ but we think the default is constant 1, then you can weight it in some way. 

Juul's concept of centrality goes back to a 2009 paper \cite{lopez2009concept}, recommending of sampling 3 curves at a time (check?). 
- Where Juul's method comes from?
- How depth handles phase variation?
- Search for functional band depth in R, check fbplot in fda package \url{https://www.rdocumentation.org/packages/fda/versions/5.1.9/topics/fbplot} how to specify the size of subensemble.
- check also \cite{sun2012exact}

- This implies that the centrality score cannot be lower than the resampled proportion. 
That is, if the size of a subensemble is $p\%$ of the whole ensemble, any curve in the subensemble will have score of 1 at least $p\%$ of the time but the centrality score cannot be less that $p$. Specifically, each curve in the whole ensemble has a $p\%$ chance to be picked in the subensemble, thus be in the envelope, so the centrality score cannot be less than the size of the subensemble.     
- Is there a upper bound? We are not sure. Thinking about a straight line with other curves are curved up or down etc.
% %%%%%%%%%%%%%%%%%%%%%%%%%%%%%%%%%%%%%%%
\subsection{Literature Review}
\citep{juul2021fixed}

\cite{probert2016decision} looking at different metrics (probes) of an epidemic.
% %%%%%%%%%%%%%%%%%%%%%%%%%%%%%%%%%%%%%%%
\subsection{emails}

{\bf Ben,  
date: Feb 9, 2021}

There are several somewhat orthogonal questions, all geared to the general question of "what is a sensible way to define a confidence interval or 'typical' set of curves out of an ensemble?

 1. Do we apply measures of centrality directly to the curves or to
probes derived from the curves?
 2.  What measure of centrality/typical-ness do we apply?
 3.  If it depends on distance, what distance do we use?
 4. Do we define centrality by difference from a central point
(centroid/Fréchet mean) or by average closeness to the rest of the
set? Does the answer differ depending on step 3 (choice of distance)?

(And finally: how do Juul et al's methods fit into this?)

Another email the same day:
you might be able to show that the conjecture does *not* hold for some fairly simple distance metric (e.g. log(Euclidean) or some similar kind of geometric-mean distance), in which case it would also settle most of the question.




\end{document}
